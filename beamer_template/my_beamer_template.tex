\documentclass[10pt,utf8]{beamer}

\mode<presentation> {
%  \usetheme{Boadilla}
  \usetheme{Madrid}
%	\usetheme{Fzu}
  \setbeamercovered{transparent}
}

\usepackage{palatino}
\usepackage{graphicx}
\usepackage{array}
\usepackage{color}
\usepackage{subfigure}
\usepackage{colortbl}
\usepackage{amsmath}
\usepackage{hyperref}
%\usepackage{tikz}
%\usetikzlibrary{arrows,shapes,backgrounds}


%\definecolor{MyDarkGreen}{rgb}{0.3,0.7,0.3}

\title{CHANGE ME: prezentation title}
\author{Vojtěch Juránek}
\institute[Red Hat]{JBoss - a division by Red Hat}
\date{CHANGE ME: date, CHANGE ME: conference name, CHANGE ME: place}

\newenvironment{mylisting}
{\begin{list}{}{\setlength{\leftmargin}{1em}}\item\scriptsize\bfseries}
{\end{list}}

\newenvironment{mytinylisting}
{\begin{list}{}{\setlength{\leftmargin}{1em}}\item\tiny\bfseries}
{\end{list}}


\begin{document}

%\tikzstyle{every picture}+=[remember picture]
%\tikzstyle{na} = [baseline=-.5ex]


\begin{frame}
 \titlepage
\end{frame}


\begin{frame}
  \frametitle{Outline}
  \begin{itemize}
    \item ???
  \end{itemize}
\end{frame}


\begin{frame}
  \frametitle{???}
\end{frame}


%%%%%%%%%%%%%%%%%%% SOME EXAMPLES %%%%%%%%%%%%%%%%%%%%%%

% \begin{frame}[fragile] %needs to be here to use verbatim
%  \textit{???} \\
%  \vspace{0.3cm}
%  \pause
%  \begin{scriptsize}\url{https://???}\end{scriptsize}
%  \scriptsize{
%    \begin{verbatim}
% wget -O /etc/yum.repos.d/jenkins.repo http://pkg.jenkins-ci.org/redhat/jenkins.repo
% rpm --import http://pkg.jenkins-ci.org/redhat/jenkins-ci.org.key
% yum -y install jenkins
% service jenkins start
%    \end{verbatim}
%  }
%   \begin{columns}
%   \column{0.65\textwidth}
%   \begin{itemize}
%   \item PHP:
% 		\begin{itemize}
% 			\item Check \url{http://jenkins-php.org/} or book \texttt{PHP projects with Jenkins} by O'Reilly
% 		\end{itemize}
%   \pause
%   \item Far not a complete list, list above is just my personal selection of some interesting plugins! Also plugins for other languages exists.
%   
%  \end{itemize}
%  \column{0.33\textwidth}
% 	\begin{figure}
% 		\centering
% 		\includegraphics[width=3cm]{./img/php_book.eps}
% 	\end{figure}
%   \end{columns}
% \end{frame}
% 



% \begin{frame}
% 	\frametitle{Releases, packages}
% 	\begin{columns}
% 		\vspace{-5cm}
% 		\column{0.59\textwidth}
% 		\url{http://jenkins-ci.org}\\
% 		\vspace{1cm}
% 		Release cycle:
% 		\begin{itemize}
% 		 \item Released usually weekly \tikz[na] \coordinate (s_weekly); - release early, release often.
% 		 \item Long term support (LTS) release \tikz[na] \coordinate (s_LTS); - every 3 months, every month minor release with backports of major bug fixes.
% 		\end{itemize}
% 		\vspace{1cm}
% 		
% 		Distribution (Jenkins is a Java servlet):
% 		\begin{itemize}
% 		 \item Web archive (WAR). \tikz[na] \coordinate (s_war);
% 		 \item Native package. \tikz[na] \coordinate (s_native);
% 		\end{itemize}
% 		\vspace{4cm}
% 
% 	
% 		\column{0.4\textwidth}
% 		\vspace{0.5cm}
% 		%\tikzstyle{background grid}=[draw, black!50,step=.5cm]
% 		\begin{tikzpicture}%[show background grid]
%             % Put the graphic inside a node. This makes it easy to place the
%             % graphic and to draw on top of it. 
%             % The above right option is used to place the lower left corner
%             % of the image at the (0,0) coordinate. 
%             \node [inner sep=0pt,above right] 
%                 {\includegraphics[width=8cm]{./img/jenkins_download.eps}};
%             % show origin
%             %\fill (0,0) circle (2pt);
%             % define destination coordinates
%             \path (0.8,10.5) coordinate (weekly)
%                   (2,10.5) coordinate (LTS)
%                   (1.5,10) coordinate (war)
%                   (0.5,7) coordinate (native);
%         \end{tikzpicture}
% 	\end{columns}
% % define overlays
% % Note the use of the overlay option. This is required when 
% % you want to access nodes in different pictures.
% \begin{tikzpicture}[overlay]
%         \path[->,red,thick] (s_weekly) edge [bend left] (weekly);
%         \path[->,blue,thick] (s_LTS) edge [bend left] (LTS);
%         %\path[->,red,thick] (s_war) edge [out=0, in=-90] (war);
%         \path[->,red,thick] (s_war) edge [out=0, in=-90] (war);
%         \path[->,blue,thick] (s_native) edge [out=0, in=-90] (native);
% \end{tikzpicture}
% 
% \end{frame}



\end{document}