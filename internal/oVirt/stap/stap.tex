\documentclass[10pt,utf8]{beamer}

\mode<presentation> {
%  \usetheme{Boadilla}
  \usetheme{Madrid}
%	\usetheme{Fzu}
  \setbeamercovered{transparent}
}

\usepackage{palatino}
\usepackage{graphicx}
\usepackage{array}
\usepackage{color}
\usepackage{subfigure}
\usepackage{colortbl}
\usepackage{amsmath}
\usepackage{hyperref}
\usepackage{listings}
\usepackage{fancyvrb}

%\setbeamertemplate{caption}{\raggedright\insertcaption\par} %turn off caption prefix ("Figure")

\title{Intorduction to systemtap}
\author{Vojtěch Juránek}
\institute[Red Hat]{oVirt storage team}
\date{3.~12.~2019}

\lstdefinestyle{Bash}{
	basicstyle          = \large\ttfamily,
	language            = Bash,
	numbers             = left,
	numberstyle         = \small,
	stepnumber          = 1,
	numbersep           = 5pt,
	backgroundcolor     = \color{white},
	showspaces          = false,
	showstringspaces    = false,
	showtabs            = false,
	frame               = single,
	tabsize             = 2,
	captionpos          = b,
	breaklines          = true,
	breakatwhitespace   = false,
	morestring          = [b]",
	stringstyle         = \color{blue},
	keywordstyle        = \color{magenta},
	commentstyle        = \color{gray},
	identifierstyle     = \color{black},
	moredelim           = **[is][\bfseries]{`}{`},
	moredelim           = **[is][\color{magenta}]{!}{!}, 
	fancyvrb            = true,
}


\begin{document}

\begin{frame}
	\titlepage
\end{frame}

\begin{frame}
	\frametitle{What is systemtap}
	\begin{itemize}
		\item Tool for probing and gathering information about running Linux system.
		\item Also scripting language for writing systemtap probes.
	\end{itemize}
\end{frame}

\begin{frame}
	\frametitle{How it works}
	\begin{itemize}
		\item Probe script is translated into C.
		\item C program is compiled into loadable kernel module.
		\item Loaded into the kernel at the beginning of the script.
		\item Unloaded from the kernel once script finishes.
	\end{itemize}
\end{frame}

\begin{frame}[fragile]
    \frametitle{Setup}
    \begin{itemize}
        \item Install \texttt{systemtap} package: \texttt{dnf install systemtap}
        \item Eventually install requested version of \texttt{kernel-devel} package if systemtap asks for it.
        \item Create "hello world" script  \texttt{hello.stp}:
            \begin{lstlisting}[style=Bash]
probe begin {
  printf("Hello world!\n")
  exit()
}
            \end{lstlisting}
        \item And run it: \texttt{stap hello.stp}
        \item You can use also shebang line in the script and make it executable: \texttt{\#!/usr/bin/env stap}
    \end{itemize}

\end{frame}


\begin{frame}
	\frametitle{Resources}
	\begin{itemize}
		\item \color{blue}\href{http://sourceware.org/systemtap/tutorial.pdf}{systemtap tutorial}
		\item \href{https://sourceware.org/systemtap/SystemTap_Beginners_Guide/}{systemtap beginner guide}
		\item \href{https://access.redhat.com/documentation/en-us/red\_hat\_enterprise_linux/7/html-single/systemtap\_beginners\_guide/index}{systemtap beginner guide, RHEL 7 documentation}
		\item \href{https://access.redhat.com/documentation/en-us/red_hat_developer\_toolset/8/html/user\_guide/chap-systemtap}{systemtap chapet in RHEL 8 User Guide}
		\item \href{https://sourceware.org/systemtap/examples/}{systemtap examples}
	\end{itemize}
	
\end{frame}

\end{document}
